

Safety, often associated with security and risk avoidance, is interpreted in diverse ways, influenced by context, culture, and individual perspectives. Recognizing its multifaceted nature is key to developing effective safety strategies.

\subsection*{Safety Carriers}

To address the ambiguity of safety, the three concepts of \textit{Risk}, \textit{Reliability}, and \textit{Resilience} are used for definition. 

\begin{itemize}
    \item \textbf{Reliability}: Defined as the probability of a system performing its intended function under specified conditions for a specified period.
    \item \textbf{Risk}: The combination of the likelihood of an event and its consequences.
    \item \textbf{Resilience}: The ability of a system to anticipate, absorb, adapt, recover and adapt after disruption to disruptions.
\end{itemize}

\subsubsection*{Evolution of Safety Science}

Safety science has evolved through distinct phases. Initially focused on technological failures, the recognition of human errors shifted the focus toward human factors. Subsequently, the perspective broadened to socio-technical systems, to address the complex interplay between human, technical, and organizational elements. The latest phase, resilience, prioritizes system adaptability and disruption management, moving from preventing specific failures to enhancing overall system robustness.

\subsubsection*{Approaches to Safety}

Safety-I and Safety-II present divergent views on system safety, with the former focusing on preventing failures and the latter on ensuring successful operations.

\begin{tabularx}{\textwidth}{X X}
\toprule
\textbf{Safety-I} & \textbf{Safety-II} \\
\midrule
FFocuses on failure states and malfunctioning. & Shifts focus to include successful hazard avoidance. \\
Binomial view of system states (functioning or malfunctioning). & Recognizes the complexity and intractability of modern systems, implying a spectrum of operational states beyond just functioning or malfunctioning. \\
Strong causality credo, emphasizing direct cause-and-effect relationships in system failures. & Emphasizes emergence over causality, acknowledging that system behaviors may arise from complex interactions not predictable by linear causality. \\
\bottomrule
\end{tabularx}

\subsection*{Risk vs. Resilience}

The concepts of risk and resilience differ in their approach to system safety. Risk assessment quantifies known threats, whereas resilience addresses adaptability to unknown or unpredictable challenges. Both concepts are integral to comprehensive safety management strategies.

\subsubsection*{Did you know - Risk as Iceberg}
\begin{mdframed}[backgroundcolor=gray!20] 
    The iceberg metaphor often used to describe the difference between risk (the visible part) and resilience (the hidden mass) illustrates the complexity and unpredictability of safety challenges. It emphasizes that what lies beneath the surface, the systems deep uncertainties requires as much attention as the visible risks.
\end{mdframed}

\begin{tabularx}{\textwidth}{X X}
\toprule
\textbf{Risk Management} & \textbf{Resilience} \\
\midrule
Focuses on quantifiable threats and known risks, aiming to prevent specific, identifiable failures. & Prepares for unknown, low-probability events, emphasizing adaptability and recovery from various disruptions. \\
Compared to the visible part of an iceberg, it represents clear and identifiable risks. & Analogous to the hidden mass of an iceberg, representing unseen and unpredictable challenges. \\
Evaluates how identified risks impact system functionality, seeking to maintain operational integrity. & Considers the system's capacity for absorption, adaptation, and recovery from disruptions, beyond mere survival. \\
Uses a high-low risk matrix to prioritize threats for targeted mitigation efforts. & Focuses on building capacities and strategies that enable the system to cope with diverse scenarios, not strictly defined by risk levels. \\
Employs a bottom-up strategy, identifying and mitigating specific risks at the operational level. & Adopts a top-down strategic approach, creating policies and frameworks that foster system-wide resilience. \\
\bottomrule
\end{tabularx}