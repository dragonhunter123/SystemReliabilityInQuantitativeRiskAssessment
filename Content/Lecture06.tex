

Quantitative Risk Assessment (QRA) represents a systematic methodology for evaluating the risks within complex systems. It encompasses the identification of potential hazards, the assessment of their probabilities, the estimation of associated risks, and the formulation of effective management strategies.

\subsection*{Hazard Identification}
The initial phase in QRA involves pinpointing potential hazards, employing various methods for comprehensive coverage:

\begin{itemize}
    \item \textbf{Failure Mode and Effects Analysis (FMEA)}: This systematic technique evaluates processes to identify possible failures and their impacts, facilitating preemptive countermeasures.
    \item \textbf{Checklists}: Leverages standardized checklists derived from historical data and industry benchmarks to recognize known hazards.
    \item \textbf{Unstructured Brainstorming}: Engages experts in freeform discussion to unearth unexpected risks, enhancing hazard identification breadth.
    \item \textbf{Preliminary Hazard Analysis (PHA)}: Conducts early-stage reviews to identify and assess potential hazards, especially valuable during project inception.
\end{itemize}

\subsection*{Probability Assessment}
Following hazard identification, the likelihood of each identified hazard is assessed using both inductive and deductive models:

\begin{itemize}
    \item Fault Trees: The probability of the top event is a function of the probabilities of the basic events, calculated using the logic represented by the gates.
    \item Event Trees: The probability of each outcome is calculated by multiplying the probabilities along the path leading to that outcome.
\end{itemize}

\subsubsection*{Fault Tree Analysis (FTA)}
\begin{itemize}
    \item \textbf{Minimum Cut Set}: Identifies the simplest combination of events leading to failure, aiding in calculating system failure probabilities.
    \item \textbf{Deductive Modeling}: Traces from a known undesirable outcome to its root causes, employing logical gates to model event interdependencies.
    \item \textbf{Logical Gates Translation}:
        \begin{itemize}
            \item \textbf{AND Gate}: \(A \land B\) - Both A and B must occur for the output event.
            \item \textbf{OR Gate}: \(A \lor B\) - Either A, B, or both must occur for the output event.
            \item \textbf{NOT Gate}: \(\lnot A\) - The output event occurs if A does not occur.
            \item \textbf{NAND Gate}: \(\lnot(A \land B)\) - The output event occurs if it's not true that both A and B occur.
            \item \textbf{NOR Gate}: \(\lnot(A \lor B)\) - The output event occurs if neither A nor B occurs.
            \item \textbf{XOR Gate}: \(A \oplus B\) - The output event occurs if either A or B, but not both, occur.
        \end{itemize}
\end{itemize}

\subsubsection*{Event Tree Analysis (ETA)}
\begin{itemize}
    \item \textbf{Initial Event}: Commences with a specific failure or hazard, exploring its subsequent effects on the system.
    \item \textbf{Inductive Modeling}: Evaluates the consequences of a component failure, mapping out possible event sequences and their outcomes.
    \item \textbf{Probability Assessment}: Determines the cumulative probability of each pathway by multiplying the probabilities of sequential events.
\end{itemize}

\noindent\begin{tabularx}{\textwidth}{X X}
\toprule
\textbf{Fault Tree Analysis (FTA)} & \textbf{Event Tree Analysis (ETA)} \\
\midrule
\textbf{Orientation:} Deductive (Top-Down) & \textbf{Orientation:} Inductive (Bottom-Up) \\
\textbf{Starting Point:} Focuses on a specific undesirable event (Top event) & \textbf{Starting Point:} Begins with an initiating event or failure \\
\textbf{Methodology:} Traces back from the top event to identify possible causes & \textbf{Methodology:} Analyzes the forward progression of events following an initiator \\
\textbf{Modeling:} Uses logic gates (AND, OR) to model the combination of events leading to the top event & \textbf{Modeling:} Maps out different scenarios (paths) leading to various outcomes based on successive events \\
\textbf{Purpose:} Aimed at understanding the causes of a specific failure & \textbf{Purpose:} Aimed at exploring the consequences of an initiating event \\
\textbf{Advantages:} Effective for identifying root causes of system failures. & \textbf{Advantages:} Excellent for understanding the dynamic progression of events following an initiating event. \\
\textbf{Disadvantages:} Can be less effective for dynamic systems where event sequences are important. & \textbf{Disadvantages:} May not identify all potential initiating events. \\
\end{tabularx}

\noindent % Ensures that we start from the very left of the text area
\begin{minipage}[t]{0.48\textwidth}
\centering
\begin{tikzpicture}[node distance=1.5cm, scale=0.8, transform shape]
\node (top) [rectangle, draw] {Top Event};
\node (and1) [circle, draw, below=of top] {AND};
\node (basic1) [rectangle, draw, below left=of and1] {Basic Event 1};
\node (basic2) [rectangle, draw, below right=of and1] {Basic Event 2};
\draw [-] (basic1) -- (and1);
\draw [-] (basic2) -- (and1);
\draw [-] (and1) -- (top);
\end{tikzpicture}
\end{minipage}%
\hfill % Fills the space between minipages
\begin{minipage}[t]{0.48\textwidth}
\centering
\begin{tikzpicture}[node distance=1.5cm, scale=0.8, transform shape, auto]
\node (init) [rectangle, draw] {Initiating Event};
\node (event1) [rectangle, draw, right=of init] {Event 1};
\node (outcome1) [rectangle, draw, above right=of event1] {Outcome 1};
\node (outcome2) [rectangle, draw, below right=of event1] {Outcome 2};
\draw [->] (init) -- (event1);
\draw [->] (event1) -- (outcome1);
\draw [->] (event1) -- (outcome2);
\end{tikzpicture}
\end{minipage}
\newline
\bottomrule

\newpage \subsubsection*{Did you know? - Bowtie Model}
\begin{mdframed}[backgroundcolor=gray!20] 
Bowtie Model in Risk Management: The bowtie model, which combines elements of both FTA and ETA, provides a visually intuitive way to understand and manage risks. By linking causes to effects through a central event, the model offers a holistic view of risk management strategies, emphasizing the importance of both prevention and mitigation.
\end{mdframed}

\subsection*{Cut Sets and Path Sets}
In the realm of Probability Assessment, particularly within the methodologies of Fault Tree Analysis (FTA) and Event Tree Analysis (ETA), the concepts of Cut Sets and Path Sets play crucial roles. These sets help in dissecting the system's reliability structure, offering insights into how component failures can lead to system-wide impacts or how certain components ensure system functionality. 

\subsubsection*{Cut set}
    \begin{itemize}
        \item Description: A combination of component failures that would lead to system failure.
        \item Importance: Identifying cut sets helps in understanding how different failures can combine to cause system failure.
        \item \textbf{Minimum Cut Set} The smallest set of failures that can cause system failure. It is useful for prioritizing which components to make more reliable to improve overall system safety.
    \end{itemize}
\subsubsection*{Path Set}
    \begin{itemize}
        \item Description: A set of components that, if operational, ensure the system's success.
        \item Application: Used to identify critical paths in a system for maintaining functionality.
        \item \textbf{Minimum Path Set}The smallest set of operational components necessary for system success. It helps in focusing on essential components for system reliability.
    \end{itemize}

\subsection*{Risk Estimation and Evaluation}
With hazards identified and their probabilities assessed, the next step involves estimating and evaluating the overall risk:

\begin{itemize}
    \item \textbf{Risk Estimation}: Calculates the frequency and severity of hazardous events.
    \item \textbf{Risk Evaluation}: Compares the estimated risks against predefined criteria to ascertain their acceptability or necessitate mitigation.
\end{itemize}

\subsection*{Risk Management}
The final step in the QRA process is risk management. This phase involves developing and implementing strategies to mitigate identified risks, as well as monitoring and reviewing the effectiveness of these strategies over time. Effective risk management not only aims to reduce the likelihood and impact of hazardous events but also ensures that the system remains robust against future risks.

\begin{itemize}
    \item \textbf{Risk Mitigation Strategies}: Develops and executes plans to reduce or eliminate risks.
    \item \textbf{Monitoring and Review}: Involves ongoing surveillance of the risk landscape and reassessment of risk management efficacy.
\end{itemize}